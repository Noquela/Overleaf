% ==================================================================
% TEMPLATE TCC MACKENZIE - BRUNO GASPARONI BALLERINI
% Baseado nas normas do Guia do TCC 2022 - Universidade Presbiteriana Mackenzie
% ==================================================================

\documentclass[12pt,a4paper,oneside]{report}

% ==================================================================
% PACOTES NECESSÁRIOS
% ==================================================================
\usepackage[utf8]{inputenc}
\usepackage[T1]{fontenc}
\usepackage[portuguese]{babel}
\usepackage{mathptmx} % Times New Roman
\usepackage{setspace}
\usepackage{geometry}
\usepackage{titlesec}
\usepackage{tocloft}
\usepackage{fancyhdr}
\usepackage{graphicx}
\usepackage{amsmath}
\usepackage{amsfonts}
\usepackage{amssymb}
\usepackage{indentfirst}
\usepackage{caption}
\usepackage{subcaption}
\usepackage{float}
\usepackage[hidelinks]{hyperref}
\usepackage{url}
\usepackage{booktabs}
\usepackage{array}
\usepackage{multirow}
\usepackage{longtable}

% ==================================================================
% CONFIGURAÇÕES GERAIS MACKENZIE
% ==================================================================

% Margens: 3cm (superior e esquerda), 2cm (inferior e direita)
\geometry{
    a4paper,
    top=3cm,
    left=3cm,
    bottom=2cm,
    right=2cm
}

% Espaçamento 1.5
\onehalfspacing

% Recuo de parágrafo 1,25cm
\setlength{\parindent}{1.25cm}

% Espaçamento entre parágrafos (1.5 conforme norma)
\setlength{\parskip}{0pt}

% ==================================================================
% CONFIGURAÇÃO DE TÍTULOS (NORMAS MACKENZIE)
% ==================================================================

% Seção primária: 1 LETRAS MAIÚSCULAS EM NEGRITO
\titleformat{\chapter}[hang]
{\normalfont\fontsize{12}{14.4}\bfseries}
{\thechapter}{1em}{\MakeUppercase}
\titlespacing*{\chapter}{0pt}{0pt}{12pt}

% Seção secundária: 1.1 LETRAS MAIÚSCULAS SEM NEGRITO
\titleformat{\section}[hang]
{\normalfont\fontsize{12}{14.4}}
{\thesection}{1em}{\MakeUppercase}
\titlespacing*{\section}{0pt}{12pt}{12pt}

% Seção terciária: 1.1.1 Letras minúsculas em negrito
\titleformat{\subsection}[hang]
{\normalfont\fontsize{12}{14.4}\bfseries}
{\thesubsection}{1em}{}
\titlespacing*{\subsection}{0pt}{12pt}{12pt}

% Seção quaternária: 1.1.1.1 Letras minúsculas sem negrito
\titleformat{\subsubsection}[hang]
{\normalfont\fontsize{12}{14.4}}
{\thesubsubsection}{1em}{}
\titlespacing*{\subsubsection}{0pt}{12pt}{12pt}

% Seção quinária: 1.1.1.1.1 Letras minúsculas em itálico
\titleformat{\paragraph}[hang]
{\normalfont\fontsize{12}{14.4}\itshape}
{\theparagraph}{1em}{}
\titlespacing*{\paragraph}{0pt}{12pt}{12pt}

% ==================================================================
% NUMERAÇÃO DE PÁGINAS
% ==================================================================
\pagestyle{fancy}
\fancyhf{}
\fancyhead[R]{\fontsize{11}{13.2}\selectfont\thepage}
\renewcommand{\headrulewidth}{0pt}

% ==================================================================
% CONFIGURAÇÃO DO SUMÁRIO
% ==================================================================
\renewcommand{\contentsname}{SUMÁRIO}
\renewcommand{\cftchapfont}{\bfseries}
\renewcommand{\cftsecfont}{\normalfont}
\renewcommand{\cftsubsecfont}{\normalfont}

% ==================================================================
% INÍCIO DO DOCUMENTO
% ==================================================================
\begin{document}

% Página de rosto não numerada
\pagenumbering{gobble}

% CAPA
% ==================================================================
% CAPA - CONFORME MODELO MACKENZIE (Apêndice D do Guia)
% ==================================================================

\thispagestyle{empty}

\begin{center}

% Espaçamento superior
\vspace*{3cm}

% Nome da instituição
{\fontsize{12}{14.4}\selectfont\bfseries\MakeUppercase{UNIVERSIDADE PRESBITERIANA MACKENZIE}}\\[0.8cm]
{\fontsize{12}{14.4}\selectfont\bfseries\MakeUppercase{Centro de Ciências e Tecnologia – CCT}}\\[0.5cm]
{\fontsize{12}{14.4}\selectfont\bfseries\MakeUppercase{Curso de Engenharia de Produção}}

\vspace{5cm}

% Nome do autor
{\fontsize{12}{14.4}\selectfont\bfseries\MakeUppercase{BRUNO GASPARONI BALLERINI}}

\vspace{5cm}

% Título do trabalho
{\fontsize{12}{14.4}\selectfont\bfseries\MakeUppercase{%
COMPARAÇÃO ENTRE MÉTODOS DE ALOCAÇÃO DE CARTEIRAS:\\[0.3cm]
MARKOWITZ, EQUAL WEIGHT E RISK PARITY\\[0.3cm] 
NO MERCADO BRASILEIRO (2018–2019)%
}}

\vfill

% Local e ano
{\fontsize{12}{14.4}\selectfont
Campinas\\[0.3cm]
2025}

\end{center}
\newpage

% FOLHA DE ROSTO
% ==================================================================
% FOLHA DE ROSTO - CONFORME MODELO MACKENZIE (Apêndice E do Guia)
% ==================================================================

\begin{center}

% Nome do autor
{\fontsize{12}{14.4}\selectfont\MakeUppercase{BRUNO GASPARONI BALLERINI}}\\
{\fontsize{12}{14.4}\selectfont RA: 10387933}\\

\vspace{4cm}

% Título do trabalho
{\fontsize{12}{14.4}\selectfont\MakeUppercase{%
COMPARAÇÃO ENTRE MÉTODOS DE ALOCAÇÃO DE CARTEIRAS:\\
MARKOWITZ, EQUAL WEIGHT E RISK PARITY\\
NO MERCADO BRASILEIRO (2018–2019)%
}}\\

\vspace{3cm}

\end{center}

% Texto da natureza do trabalho (alinhado à direita, a partir do meio da página)
\begin{flushright}
\begin{minipage}{8cm}
\fontsize{11}{13.2}\selectfont
\setlength{\parindent}{0cm}
\setlength{\parskip}{0pt}

Trabalho de Conclusão de Curso apresentado ao Curso de Engenharia de Produção da Universidade Presbiteriana Mackenzie -- Campus Campinas, como requisito parcial para obtenção do título de Engenheiro de Produção.

\vspace{1cm}

Orientador: Prof. Dr. Ricardo Antonio Fernandes

\end{minipage}
\end{flushright}

\vfill

\begin{center}
% Local e ano
{\fontsize{12}{14.4}\selectfont
Campinas\\
2025}
\end{center}
\newpage

% Inicia numeração romana para pré-textuais
\pagenumbering{roman}
\setcounter{page}{1}

% LISTA DE FIGURAS
% ==================================================================
% LISTA DE FIGURAS
% ==================================================================

\chapter*{LISTA DE FIGURAS}
\addcontentsline{toc}{chapter}{LISTA DE FIGURAS}

\vspace{1cm}

\noindent
Figura 1 -- Fluxograma da Metodologia \dotfill 16

% Adicione outras figuras conforme necessário no formato:
% Figura X -- Título da figura \dotfill página
\newpage

% LISTA DE TABELAS
% ==================================================================
% LISTA DE TABELAS
% ==================================================================

\chapter*{LISTA DE TABELAS}
\addcontentsline{toc}{chapter}{LISTA DE TABELAS}

\vspace{1cm}

\noindent
Tabela 1 -- Estudos Correlatos \dotfill 8\\
Tabela 2 -- Etapas da Pesquisa e Ferramentas Utilizadas \dotfill 17

% Adicione outras tabelas conforme necessário
\newpage

% LISTA DE ABREVIATURAS E SIGLAS
% ==================================================================
% LISTA DE ABREVIATURAS E SIGLAS
% ==================================================================

\chapter*{LISTA DE ABREVIATURAS E SIGLAS}
\addcontentsline{toc}{chapter}{LISTA DE ABREVIATURAS E SIGLAS}

\vspace{1cm}

\noindent
API -- Application Programming Interface\\
B3 -- Brasil Bolsa Balcão\\
CDI -- Certificado de Depósito Interbancário\\
CVM -- Comissão de Valores Mobiliários\\
IBOV -- Índice Bovespa\\
ML -- Machine Learning\\
PIB -- Produto Interno Bruto\\
TCC -- Trabalho de Conclusão de Curso\\
VIX -- Volatility Index
\newpage

% LISTA DE FÓRMULAS
% ==================================================================
% LISTA DE FÓRMULAS
% ==================================================================

\chapter*{LISTA DE FÓRMULAS}
\addcontentsline{toc}{chapter}{LISTA DE FÓRMULAS}

\vspace{1cm}

\noindent
Fórmula 1 -- Cálculo do peso no modelo Risk Parity \dotfill 175\\
Fórmula 2 -- Índice de Sharpe \dotfill 176\\
Fórmula 3 -- Sortino Ratio \dotfill 177
\newpage

% RESUMO
% ==================================================================
% RESUMO
% ==================================================================

\chapter*{RESUMO}
\addcontentsline{toc}{chapter}{RESUMO}

\vspace{1cm}

Este trabalho tem como objetivo comparar o desempenho de três métodos de alocação de carteiras --- Markowitz, Equal Weight e Risk Parity --- utilizando dados de ativos da B3 no período de 2018 a 2019. Para a avaliação das carteiras, foram empregados o Índice de Sharpe, que mede o retorno ajustado ao risco total, e o Sortino Ratio, que considera apenas a volatilidade negativa, focando nos riscos de perda. O estudo adota uma abordagem quantitativa, descritiva e comparativa, utilizando ferramentas computacionais para otimização e análise. Os resultados pretendem oferecer insights relevantes para investidores em contextos de elevada volatilidade e incerteza, como o mercado brasileiro.

\vspace{0.5cm}

\noindent
\textbf{Palavras-chave:} Alocação de Carteiras; Markowitz; Equal Weight; Risk Parity; Índice de Sharpe; Sortino Ratio.
\newpage

% ABSTRACT
% ==================================================================
% ABSTRACT
% ==================================================================

\chapter*{ABSTRACT}
\addcontentsline{toc}{chapter}{ABSTRACT}

\vspace{1cm}

This study aims to compare the performance of three portfolio allocation methods --- Markowitz, Equal Weight, and Risk Parity --- using B3 asset data from 2018 to 2019. Portfolio evaluation employed the Sharpe Ratio, which measures return adjusted for total risk, and the Sortino Ratio, focusing specifically on downside risk. The study adopts a quantitative, descriptive, and comparative approach, utilizing computational tools for portfolio optimization and performance analysis. The results aim to provide relevant insights for investors operating in high volatility markets such as Brazil.

\vspace{0.5cm}

\noindent
\textbf{Keywords:} Portfolio Allocation; Markowitz; Equal Weight; Risk Parity; Sharpe Ratio; Sortino Ratio.
\newpage

% SUMÁRIO
\tableofcontents
\newpage

% Inicia numeração arábica para textuais
\pagenumbering{arabic}
\setcounter{page}{1}

% ELEMENTOS TEXTUAIS
% ==================================================================
% 1 INTRODUÇÃO
% ==================================================================

\chapter{INTRODUÇÃO}

A alocação de ativos é amplamente reconhecida como um dos principais determinantes do desempenho de carteiras de investimento. Estudos clássicos, como o de \cite{brinson1986determinants}, indicam que mais de 90\% da variância do retorno de uma carteira pode ser explicada por decisões de alocação estratégica de ativos, superando o impacto da seleção individual de ativos ou do timing de mercado.

Nesse contexto \cite{markowitz1952portfolio} propôs o modelo de Média-Variância, que representa um marco na teoria moderna de portfólios ao formalizar matematicamente a relação entre risco e retorno esperado. Sua abordagem busca identificar combinações eficientes de ativos que maximizem o retorno esperado para um dado nível de risco, ou minimizem o risco para um retorno desejado.

Em ambientes caracterizados por elevada volatilidade e incerteza, como frequentemente ocorre em mercados emergentes, a definição de uma estratégia de alocação eficiente torna-se ainda mais desafiadora, exigindo metodologias que consigam lidar com instabilidade, correlações variáveis e estimativas imperfeitas de risco e retorno \cite{ilmanen2022investing}.

Entre as metodologias mais conhecidas e aplicadas na literatura acadêmica e no mercado estão o modelo de Média-Variância, proposto por Markowitz, a estratégia de alocação por pesos iguais (Equal Weight) e a metodologia de paridade de risco (Risk Parity). Cada uma dessas abordagens apresenta características específicas, vantagens próprias e limitações que precisam ser cuidadosamente analisadas em ambientes voláteis.

O modelo de \cite{markowitz1952portfolio} revolucionou a teoria financeira ao formalizar matematicamente a construção de carteiras eficientes, baseando-se na relação entre risco e retorno esperado. Seu principal objetivo é identificar a combinação ótima de ativos que maximize o retorno esperado para um nível específico de risco ou minimize o risco para determinado nível de retorno. Entretanto, esse modelo assume condições como a normalidade dos retornos dos ativos e a estabilidade das estimativas utilizadas, premissas que nem sempre se verificam na prática, especialmente em períodos de alta volatilidade ou crises financeiras \cite{michalak2024equal}.

Como alternativa de implementação mais simples, a estratégia Equal Weight distribui o capital igualmente entre todos os ativos selecionados na carteira, sem a necessidade de previsões complexas. Essa abordagem demonstra, em muitos estudos, ser bastante robusta em cenários de alta incerteza, apresentando desempenho comparável, ou até superior, a estratégias de otimização mais sofisticadas, especialmente em análises fora da amostra \cite{demiguel2009optimal}. Por outro lado, sua simplicidade implica limitações, pois ignora características fundamentais dos ativos, como volatilidade e correlação, o que pode levar a concentrações de risco inadvertidas.

A metodologia de Risk Parity, por sua vez, busca uma distribuição mais equilibrada do risco total da carteira, atribuindo menores pesos a ativos mais voláteis e maiores pesos a ativos menos voláteis. Tal abordagem vem ganhando destaque nos últimos anos por produzir carteiras mais estáveis e menos suscetíveis a erros de estimativa, com desempenho sólido em diferentes cenários econômicos \cite{maillard2010properties,palit2024study}.

No cenário brasileiro, o período compreendido entre 2016 e 2019 foi marcado por alta volatilidade no mercado acionário, com o desvio-padrão anualizado dos retornos do Ibovespa oscilando entre 20\% e 25\% \cite{gregorio2020volatilidade}. Particularmente, os anos de 2018 e 2019 coincidiram com um contexto de incerteza política e financeira, principalmente em função das eleições presidenciais e das alterações no ambiente econômico subsequente. Estudos de \cite{pereira2021impacto} mostraram que choques políticos influenciaram diretamente os retornos de ações brasileiras, especialmente de empresas com vínculos governamentais, enquanto \cite{carnahan2020electoral} comprovaram que eleições em mercados emergentes tendem a aumentar significativamente a volatilidade dos ativos no curto prazo.

Diante desse contexto de instabilidade e alta incerteza, o presente trabalho propõe uma análise comparativa entre as três estratégias de alocação de carteira --- Markowitz, Equal Weight e Risk Parity --- utilizando dados de ativos negociados na B3 no período de 2018 a 2019. A comparação do desempenho será realizada com base em duas métricas amplamente reconhecidas na literatura financeira: o Índice de Sharpe, que avalia o retorno ajustado ao risco total da carteira, e o Sortino Ratio, que considera apenas os riscos de perdas.

Com essa abordagem, pretende-se contribuir para a identificação de estratégias de alocação mais eficientes no contexto brasileiro, gerando insights relevantes tanto para investidores quanto para gestores de recursos que buscam maximizar o retorno ajustado ao risco em ambientes de elevada volatilidade e imprevisibilidade.

\section{OBJETIVO GERAL}

Analisar comparativamente o desempenho das estratégias de alocação de carteira Markowitz, Equal Weight e Risk Parity no mercado brasileiro, utilizando dados de ativos da B3 entre 2018 e 2019, com base nos indicadores Índice de Sharpe e Sortino Ratio, a fim de identificar a estratégia mais eficiente em termos de retorno ajustado ao risco.

\section{OBJETIVOS ESPECÍFICOS}

\begin{itemize}
    \item Selecionar uma amostra de aproximadamente 10 ações da B3, considerando critérios de liquidez, representatividade setorial e capitalização de mercado.
    
    \item Calcular os retornos históricos dos ativos selecionados, estimar parâmetros como médias, volatilidades e covariâncias dos retornos.
    
    \item Implementar as três estratégias de alocação (Markowitz, Equal Weight e Risk Parity), programaticamente, por meio de ferramentas computacionais.
    
    \item Realizar o rebalanceamento semestral das carteiras durante o período de 2018 a 2019.
    
    \item Calcular os Índices de Sharpe e Sortino para cada carteira e para o período consolidado.
    
    \item Comparar os desempenhos obtidos, avaliando a eficiência de cada estratégia em ambientes de alta volatilidade e instabilidade política.
\end{itemize}

\section{JUSTIFICATIVA}

A escolha da estratégia de alocação de ativos é um dos fatores mais determinantes para o desempenho de carteiras de investimento, especialmente em mercados caracterizados por alta volatilidade e incerteza, como o brasileiro. Nesse cenário, compreender a eficácia das diferentes metodologias de construção de portfólios torna-se crucial para investidores e gestores que buscam maximizar retornos e mitigar riscos.

O modelo de Markowitz, embora amplamente consolidado, apresenta limitações práticas, sobretudo pela dependência de estimativas instáveis em momentos de elevada volatilidade. A estratégia Equal Weight, por sua vez, oferece simplicidade operacional e robustez frente a erros de previsão, mas desconsidera características fundamentais dos ativos, como a volatilidade individual. Já a metodologia de Risk Parity surge como uma alternativa promissora, ao equilibrar a contribuição de risco entre os ativos, proporcionando maior estabilidade às carteiras.

No contexto do mercado brasileiro, o período de 2018 a 2019 foi particularmente marcado por oscilações intensas, impulsionadas por fatores políticos e econômicos. Segundo \cite{gregorio2020volatilidade}, a volatilidade anualizada do Ibovespa em 2018 alcançou aproximadamente 25\%, superando a média histórica observada em períodos de maior estabilidade. Esse ambiente volátil oferece uma oportunidade ideal para investigar como diferentes abordagens de alocação se comportam frente a choques exógenos e incertezas sistêmicas.

A presente pesquisa justifica-se, portanto, pela necessidade de avaliar, com dados reais e métricas consolidadas --- como o Índice de Sharpe e o Sortino Ratio ---, qual das três estratégias de alocação proporciona melhor relação risco-retorno em um mercado emergente. Além disso, os resultados obtidos podem oferecer insights relevantes para a prática de gestão de portfólios, especialmente por fundos multimercado, gestoras de ativos e investidores individuais que operam em ambientes instáveis.

Por fim, ao abordar simultaneamente indicadores clássicos e medidas voltadas ao risco de perdas, o estudo busca apresentar uma avaliação mais completa da eficiência das estratégias analisadas, contribuindo tanto para o avanço da literatura quanto para a tomada de decisão no mercado financeiro.
% ==================================================================
% 2 REFERENCIAL TEÓRICO
% ==================================================================

\chapter{REFERENCIAL TEÓRICO}

\section{MODELO DE MARKOWITZ (MÉDIA-VARIÂNCIA)}

O modelo de Média-Variância, desenvolvido por Markowitz (1952), representa um marco teórico na construção de carteiras eficientes, sendo uma das bases fundamentais da moderna teoria de investimentos. O objetivo central da metodologia é encontrar a combinação ótima de ativos que maximize o retorno esperado para um dado nível de risco ou, alternativamente, minimize o risco para um retorno esperado específico.

O modelo assume que os retornos dos ativos seguem uma distribuição normal e que os investidores são avessos ao risco, preferindo carteiras com menor volatilidade para retornos equivalentes. A construção da ``fronteira eficiente'' baseia-se na análise da média e variância dos retornos dos ativos, bem como nas covariâncias entre eles. Apesar de sua elegância teórica, o modelo enfrenta críticas, especialmente em ambientes de alta volatilidade, pela dependência excessiva de estimativas de parâmetros que podem se mostrar instáveis no tempo \cite{michalak2024equal}.

\section{ESTRATÉGIA EQUAL WEIGHT (PESOS IGUAIS)}

A estratégia Equal Weight consiste na alocação igualitária do capital entre todos os ativos da carteira, atribuindo o mesmo peso percentual para cada ativo, independentemente de suas características individuais. Essa abordagem se destaca pela simplicidade operacional e pela robustez frente a erros de previsão de retorno e volatilidade \cite{demiguel2009optimal}.

Estudos indicam que, em muitos casos, o desempenho de carteiras Equal Weight pode superar o de métodos mais sofisticados, especialmente fora da amostra. No entanto, a ausência de ajustes baseados em volatilidade ou correlação pode resultar em carteiras com concentração de riscos indesejados, especialmente em ativos mais voláteis.

\section{ESTRATÉGIA RISK PARITY (PARIDADE DE RISCO)}

A estratégia Risk Parity surgiu como uma alternativa para endereçar o problema da concentração de risco observado em abordagens tradicionais. Nessa metodologia, o objetivo é equilibrar a contribuição de risco de cada ativo no portfólio, atribuindo pesos inversamente proporcionais à volatilidade dos ativos \cite{maillard2010properties}.

O cálculo dos pesos é realizado a partir da seguinte fórmula:

\begin{equation}
w_i = \frac{(1/\sigma_i)}{\sum_{j=1}^{n}(1/\sigma_j)}
\end{equation}

onde $w_i$ representa o peso do ativo $i$ e $\sigma_i$ é o desvio-padrão dos retornos do ativo $i$. Essa abordagem tende a produzir carteiras mais estáveis e menos sensíveis a erros de estimativa, o que a torna atrativa em ambientes de alta incerteza \cite{palit2024study}.

\section{MÉTRICAS DE AVALIAÇÃO: ÍNDICE DE SHARPE E SORTINO RATIO}

A avaliação de desempenho das carteiras será baseada em duas métricas amplamente reconhecidas:

\textbf{Índice de Sharpe:} mede o retorno excedente em relação à taxa livre de risco por unidade de volatilidade total dos retornos da carteira.

\begin{equation}
\text{Sharpe} = \frac{R_p - R_f}{\sigma_p}
\end{equation}

onde $R_p$ é o retorno médio da carteira, $R_f$ é a taxa livre de risco e $\sigma_p$ é o desvio-padrão dos retornos da carteira.

\textbf{Sortino Ratio:} similar ao Sharpe Ratio, mas considera apenas a volatilidade negativa (retornos abaixo de um objetivo ou taxa mínima desejada).

\begin{equation}
\text{Sortino} = \frac{R_p - T}{\sigma_-}
\end{equation}

onde $T$ é a taxa mínima de retorno e $\sigma_-$ é o desvio-padrão dos retornos abaixo dessa taxa.

Essas métricas oferecem uma visão abrangente da relação risco-retorno, considerando tanto a variabilidade geral quanto o risco específico de perdas.

\section{PERÍODO DO ESTUDO}

A escolha do período de 2018 a 2019 para a análise comparativa entre as estratégias de alocação de carteiras --- Markowitz, Equal Weight e Risk Parity --- não foi aleatória, mas sim fundamentada em características peculiares do cenário econômico e político brasileiro. Esses dois anos representam um momento de elevada volatilidade no mercado de capitais, impulsionado principalmente pelas eleições presidenciais de 2018 e pelas subsequentes incertezas sobre a condução da política econômica do novo governo.

Durante esse intervalo, o Índice Bovespa apresentou oscilações significativas, refletindo o humor dos investidores diante de um ambiente instável e frequentemente imprevisível. Segundo \cite{gregorio2020volatilidade}, o desvio-padrão anualizado dos retornos do Ibovespa chegou a ultrapassar 25\% em determinados momentos, reforçando a natureza volátil do período.

Além do fator político, o cenário macroeconômico brasileiro ainda carregava resquícios da recessão econômica que atingiu o país entre 2014 e 2016. A lenta recuperação do Produto Interno Bruto (PIB), as reformas estruturais em discussão (como a reforma da Previdência) e as oscilações no câmbio e nas taxas de juros também contribuíram para um ambiente de incerteza que afeta diretamente as decisões de alocação de ativos.

Em mercados emergentes como o Brasil, eventos políticos têm impacto amplificado sobre os ativos financeiros, como apontado por \cite{carnahan2020electoral}, que analisaram a influência de eleições sobre a volatilidade dos mercados latino-americanos. Esse contexto adverso justifica plenamente a aplicação de metodologias de alocação que busquem eficiência mesmo em cenários instáveis, como é o caso das abordagens comparadas neste trabalho.

\section{ESTUDOS RELACIONADOS}

Diversos estudos prévios abordaram comparações entre diferentes estratégias de alocação de ativos, tanto em mercados desenvolvidos quanto emergentes. Essa revisão tem como objetivo situar a presente pesquisa dentro da literatura existente, evidenciando a relevância e originalidade do estudo.

A seguir, apresenta-se uma síntese dos principais trabalhos relacionados:

\begin{table}[h]
\centering
\caption{Estudos Correlatos}
\begin{tabular}{|p{3cm}|p{4cm}|p{3cm}|p{4cm}|}
\hline
\textbf{Autor/Ano} & \textbf{Objetivo} & \textbf{Metodologia} & \textbf{Principais Resultados} \\
\hline
\cite{demiguel2009optimal} & Comparar estratégias de otimização vs. diversificação ingênua & Análise empírica com dados de mercados desenvolvidos & Equal weight supera estratégias otimizadas fora da amostra \\
\hline
\cite{maillard2010properties} & Analisar propriedades de carteiras Risk Parity & Modelagem matemática e testes empíricos & Risk Parity oferece melhor relação risco-retorno \\
\hline
\cite{michalak2024equal} & Comparar Equal Weight vs. Hierarchical Risk Parity & Estudo comparativo com dados europeus & HRP supera EW em períodos de alta volatilidade \\
\hline
\end{tabular}
\label{tab:estudos_correlatos}
\end{table}

\section{FERRAMENTAS COMPUTACIONAIS}

A implementação das estratégias de alocação de carteiras neste trabalho será realizada com auxílio da linguagem de programação Python, amplamente reconhecida por sua flexibilidade e pelo vasto ecossistema de bibliotecas aplicadas à ciência de dados e finanças.

Entre as bibliotecas previstas, destacam-se:

\textbf{Pandas:} será utilizada para manipulação e análise de dados tabulares, como séries históricas de preços e retornos. Essa biblioteca é amplamente adotada em estudos empíricos por sua eficiência na estruturação de dados financeiros.

\textbf{NumPy:} será empregada para cálculos vetoriais e matriciais, como a média dos retornos, o desvio-padrão e, principalmente, o cálculo da matriz de covariância entre os ativos. O NumPy é a base para operações matemáticas eficientes em Python.

\textbf{cvxpy:} biblioteca voltada para otimização convexa, será utilizada para implementar a carteira de Markowitz. A ferramenta permite a formulação de problemas de programação quadrática, muito utilizada em finanças quantitativas.

\textbf{matplotlib e seaborn:} serão usadas para gerar gráficos e visualizações como a evolução das carteiras, boxplots de retorno e heatmaps de correlação, contribuindo para a análise visual dos resultados.

\textbf{yfinance:} será utilizada para a extração automática dos preços históricos ajustados dos ativos diretamente do Yahoo Finance, permitindo uma coleta de dados prática e atualizada. A biblioteca é utilizada por diversos estudos empíricos recentes.

Essas ferramentas foram escolhidas por serem de código aberto, amplamente documentadas e reconhecidas em estudos da área de finanças computacionais. A opção pelo uso de programação, em vez de planilhas, visa garantir maior precisão, replicabilidade e flexibilidade na análise. Além disso, o domínio dessas ferramentas reflete competências valorizadas no mercado financeiro, alinhando-se às demandas contemporâneas por análise quantitativa de investimentos.

\section{ANÁLISE DOS ELEMENTOS DAS FÓRMULAS}

A seguir estão descritos em detalhe todos os componentes de cada fórmula utilizada na avaliação de desempenho e construção de carteiras neste estudo:

\subsection{Fórmula 1 -- Peso na Estratégia Risk Parity}

\begin{equation}
w_i = \frac{1/\sigma_i}{\sum_{j=1}^{n}(1/\sigma_j)}
\end{equation}

Em que:
\begin{itemize}
    \item $w_i$: peso do ativo $i$ na carteira;
    \item $\sigma_i$: desvio-padrão dos retornos do ativo $i$.
\end{itemize}

\textbf{Comentário:} ao atribuir ao peso o inverso da volatilidade, ativos mais voláteis recebem menor participação, equilibrando a contribuição de risco de cada ativo.

\subsection{Fórmula 2 -- Índice de Sharpe}

\begin{equation}
\text{Sharpe} = \frac{R_p - R_f}{\sigma_p}
\end{equation}

Em que:
\begin{itemize}
    \item $R_p$: retorno médio da carteira;
    \item $R_f$: taxa livre de risco;
    \item $\sigma_p$: desvio-padrão dos retornos da carteira.
\end{itemize}

\textbf{Comentário:} o Sharpe Ratio mede quanto retorno excedente cada unidade de risco total consegue gerar.

\subsection{Fórmula 3 -- Sortino Ratio}

\begin{equation}
\text{Sortino} = \frac{R_p - T}{\sigma_-}
\end{equation}

Em que:
\begin{itemize}
    \item $R_p$: retorno médio da carteira;
    \item $T$: retorno mínimo aceitável (threshold);
    \item $\sigma_-$: desvio-padrão dos retornos abaixo de $T$.
\end{itemize}

\textbf{Comentário:} diferente do Sharpe, o Sortino considera apenas a volatilidade negativa, focando no risco de perdas.
% ==================================================================
% 3 METODOLOGIA
% ==================================================================

\chapter{METODOLOGIA}

\section{TIPO DE PESQUISA E ESTRATÉGIA METODOLÓGICA}

Este estudo é de natureza quantitativa, descritiva e comparativa, com foco na avaliação do desempenho de diferentes estratégias de alocação de ativos financeiros. A pesquisa adota uma abordagem empírica, utilizando dados históricos do mercado financeiro brasileiro para a construção e análise das carteiras.

\section{PERÍODO E AMBIENTE DE ESTUDO}

O horizonte temporal da análise compreende o período de janeiro de 2018 a dezembro de 2019, um momento de alta volatilidade e instabilidade política no Brasil. O ambiente de estudo é a B3 -- Brasil Bolsa Balcão, principal bolsa de valores brasileira.

\section{SELEÇÃO DOS ATIVOS}

A seleção dos ativos da amostra será baseada nos seguintes critérios:

\begin{itemize}
    \item \textbf{Alta liquidez histórica:} Volume médio diário de negociação superior a R\$ 50 milhões, calculado com base no histórico de janeiro de 2016 a dezembro de 2017;
    
    \item \textbf{Diversificação setorial:} Inclusão de ações de diferentes setores da economia;
    
    \item \textbf{Capitalização de mercado:} Preferência por empresas de maior valor de mercado, geralmente pertencentes ao índice Ibovespa.
\end{itemize}

Essa metodologia de seleção busca garantir que os ativos escolhidos sejam representativos, líquidos e compatíveis com práticas profissionais de alocação de portfólio, evitando viés de seleção retrospectivo.

\section{COLETA E TRATAMENTO DOS DADOS}

Os dados históricos de preços ajustados dos ativos serão coletados utilizando APIs financeiras (como Yahoo Finance), via Python.

O tratamento dos dados inclui:

\begin{itemize}
    \item Cálculo dos retornos mensais e anualizados;
    \item Estimativa das volatilidades individuais;
    \item Construção da matriz de covariância entre os retornos dos ativos;
    \item Definição da taxa livre de risco como o CDI médio anualizado do período.
\end{itemize}

\section{CONSTRUÇÃO DAS CARTEIRAS}

Serão implementadas três estratégias de alocação:

\subsection{Markowitz (Média-Variância)}
Otimização para maximizar o Índice de Sharpe, com restrições de soma dos pesos igual a 1 e ausência de vendas a descoberto.

\subsection{Equal Weight}
Alocação igualitária do capital entre os ativos.

\subsection{Risk Parity}
Alocação baseada na contribuição igual de risco de cada ativo, utilizando a fórmula:

\begin{equation}
w_i = \frac{(1/\sigma_i)}{\sum_{j=1}^{n}(1/\sigma_j)}
\end{equation}

As carteiras serão construídas usando a linguagem Python, com bibliotecas como pandas, NumPy e cvxpy.

\section{REBALANCEAMENTO DAS CARTEIRAS}

O rebalanceamento das carteiras será realizado de forma semestral, nos meses de janeiro e julho, utilizando as novas estimativas de retornos e covariâncias disponíveis em cada momento.

Custos de transação, impostos e slippage não serão considerados, representando uma limitação reconhecida da pesquisa.

\section{AVALIAÇÃO DE DESEMPENHO}

O desempenho das carteiras será avaliado por:

\begin{itemize}
    \item \textbf{Índice de Sharpe:} Avaliação do retorno excedente ajustado pela volatilidade total.
    \item \textbf{Sortino Ratio:} Avaliação do retorno excedente ajustado apenas pelo risco de perdas (volatilidade negativa).
\end{itemize}

As métricas serão calculadas:
\begin{itemize}
    \item Para cada semestre individualmente;
    \item E para o período consolidado 2018--2019.
\end{itemize}

\section{FLUXO DE PROCESSAMENTO E ANÁLISE DE RESULTADOS}

Este trabalho adota um fluxo integrado que vai da extração dos dados até a análise comparativa dos resultados, organizado em quatro fases principais:

\subsection{Coleta e Tratamento de Dados}

Inicialmente, extraem-se automaticamente as cotações ajustadas diárias dos ativos selecionados (B3, 2018--2019). Essas séries passam por validação --- remoção de duplicatas, interpolação de eventuais lacunas e filtragem de outliers ---, garantindo uma base limpa e contínua. Em sequência, convertem-se as variações diárias em retornos mensais compostos, produzindo uma tabela onde cada linha representa o retorno mensal de cada ativo.

\subsection{Cálculo de Insumos para Alocação}

A partir dos retornos mensais, calcula-se o vetor de retorno médio ($\mu$) e a matriz de covariância ($\Sigma$). Esses insumos são imediatamente utilizados em cada data de rebalanceamento (1º de janeiro e 1º de julho), garantindo que a alocação reflita apenas as informações disponíveis até aquele ponto.

\subsection{Construção de Carteiras e Extração de Pesos}

\begin{itemize}
    \item \textbf{Markowitz (Sharpe máximo):} otimiza-se a relação entre retorno excedente e risco total, sujeita a soma unitária e não negatividade dos pesos.
    
    \item \textbf{Equal Weight:} atribui-se peso igual a todos os ativos (1/N), servindo como benchmark simples.
    
    \item \textbf{Risk Parity:} ajusta-se iterativamente os pesos até que cada ativo contribua igualmente para o risco total da carteira (produto do peso pela volatilidade marginal).
\end{itemize}

Esses vetores de pesos passam a compor três séries de evolução de portfólio.

\subsection{Cálculo de Indicadores e Análise Comparativa}

As séries de retorno de cada carteira alimentam o cálculo das principais métricas:

\begin{itemize}
    \item Sharpe Ratio (retorno excedente $\div$ volatilidade total);
    \item Sortino Ratio (retorno excedente $\div$ volatilidade negativa);
    \item Drawdown Máximo (pior queda acumulada);
    \item Volatilidade Realizada (desvio-padrão efetivo dos retornos).
\end{itemize}

Para cada semestre e para o acumulado de 2018--2019, organizam-se tabelas comparativas, seguidas de:

\begin{itemize}
    \item Gráficos de linha mostrando a evolução do capital investido, permitindo visualizar divergências de performance;
    \item Box-plots das distribuições mensais de retorno, evidenciando dispersão e assimetrias;
    \item Heatmaps de correlação periódicos para observar mudanças no relacionamento entre ativos e seu impacto nas carteiras.
\end{itemize}

Na redação do TCC, cada indicador será apresentado em seção própria de Resultados, com interpretação centrada em:

\begin{itemize}
    \item Diferenças de eficiência entre as três estratégias em ambientes de alta volatilidade;
    \item Robustez dos resultados a diferentes janelas de rebalanceamento;
    \item Significância prática das variações de Sharpe e Sortino, discutindo potenciais custos de transação e limitações não consideradas.
\end{itemize}

Esse fluxo completo --- da base de preços até a síntese crítica dos resultados --- serve de alicerce para a discussão final, embasando recomendações e apontando direções para pesquisas futuras.

\section{FLUXOGRAMA METODOLÓGICO}

\begin{figure}[h]
\centering
\caption{Fluxograma da Metodologia}
% Adicionar figura quando disponível
\textit{Fonte: Elaborado pelo autor.}
\label{fig:fluxograma_metodologia}
\end{figure}

\begin{table}[h]
\centering
\caption{Etapas da Pesquisa e Ferramentas Utilizadas}
\begin{tabular}{|p{4cm}|p{4cm}|p{4cm}|}
\hline
\textbf{Etapa} & \textbf{Descrição} & \textbf{Ferramentas} \\
\hline
Coleta de Dados & Extração de preços históricos da B3 & Python, yfinance \\
\hline
Tratamento de Dados & Cálculo de retornos e estatísticas & pandas, NumPy \\
\hline
Construção de Carteiras & Implementação das três estratégias & cvxpy, pandas \\
\hline
Análise de Desempenho & Cálculo de métricas e comparação & NumPy, matplotlib \\
\hline
Visualização & Gráficos e tabelas comparativas & matplotlib, seaborn \\
\hline
\end{tabular}
\label{tab:etapas_pesquisa}
\end{table}

\section{LIMITAÇÕES DO ESTUDO}

Apesar do rigor metodológico adotado, este estudo apresenta algumas limitações que devem ser consideradas na análise dos resultados. Em primeiro lugar, não foram incorporados custos de transação, taxas, slippage e tributação nas operações de compra e venda dos ativos, o que pode gerar divergências entre os retornos simulados e os efetivamente obtidos na prática. Além disso, a utilização de séries históricas de retornos pressupõe que padrões passados se mantenham representativos para o futuro, o que pode não se confirmar em mercados sujeitos a choques exógenos e mudanças estruturais. Outra limitação refere-se à escolha de apenas três estratégias de alocação, desconsiderando alternativas mais recentes como o modelo de Hierarchical Risk Parity ou abordagens baseadas em Machine Learning, que poderiam trazer novas perspectivas. Por fim, a definição da taxa livre de risco como o CDI médio anualizado simplifica a realidade de investimentos no Brasil, que apresenta múltiplos instrumentos de renda fixa com diferentes graus de risco e liquidez. Tais limitações, embora não invalidem os resultados, indicam caminhos para aprofundamentos em pesquisas futuras.

% REFERÊNCIAS
% ==================================================================
% REFERÊNCIAS BIBLIOGRÁFICAS
% ==================================================================

\chapter*{REFERÊNCIAS BIBLIOGRÁFICAS}
\addcontentsline{toc}{chapter}{REFERÊNCIAS BIBLIOGRÁFICAS}

\vspace{1cm}

% Configuração para referências (substitua por arquivo .bib se necessário)

\begin{thebibliography}{99}

\bibitem{agrawal2018cvxpy}
AGRAWAL, A.; BOYD, S.; BUSSETI, F. CVXPY: a Python-embedded modeling language for convex optimization. \textit{Journal of Machine Learning Research}, v. 17, p. 1--5, 2018. Disponível em: https://jmlr.org/papers/volume17/16-403/16-403.pdf. Acesso em: 1 maio 2025.

\bibitem{b3_2018}
B3 -- Brasil, Bolsa, Balcão. Relatório mensal IBOB-VIX -- Outubro 2018. São Paulo: B3, 2018. Disponível em: https://www.b3.com.br/data/files/9E/97/23/7F/8AF637109A6B9155AC0D8AA8/BOLETIM\_IBOBVIX\_out2018.pdf. Acesso em: 29 abril 2025.

\bibitem{brinson1986determinants}
BRINSON, G. P.; HOOD, L. R.; BEEBOWER, G. L. Determinants of portfolio performance. \textit{Financial Analysts Journal}, v. 42, n. 4, p. 39--44, 1986. Disponível em: https://www.cfainstitute.org/-/media/documents/article/faj/1986/faj-v42-n4-39.ashx. Acesso em: 29 abril 2025.

\bibitem{carnahan2020electoral}
CARNAHAN, D.; SAIEGH, S. Electoral uncertainty and financial volatility: evidence from two-round presidential races in emerging markets. \textit{Economics and Politics}, v. 33, n. 1, p. 109--132, 2020. Disponível em: https://onlinelibrary.wiley.com/doi/abs/10.1111/ecpo.12157. Acesso em: 29 abril 2025.

\bibitem{chen2020financial}
CHEN, L.; HUANG, J. \textit{Financial Data Analysis Using Python}. Cham: Springer, 2020. Disponível em: https://link.springer.com/book/10.1007/978-3-030-57908-9. Acesso em: 1 maio 2025.

\bibitem{cvm2018}
COMISSÃO DE VALORES MOBILIÁRIOS (CVM). Boletim de Riscos -- maio 2018. Brasília: CVM, 2018. Disponível em: https://conteudo.cvm.gov.br/export/sites/cvm/estudos/analisederisco/anexos/Boletim\_Riscos\_2018-05.pdf. Acesso em: 29 abril 2025.

\bibitem{demiguel2009optimal}
DE MIGUEL, V.; GARLAPPI, L.; UPPAL, R. Optimal versus naïve diversification: how inefficient is the 1/N portfolio strategy? \textit{Review of Financial Studies}, v. 22, n. 5, p. 1915--1953, 2009. Disponível em: https://academic.oup.com/rfs/article/22/5/1915/1598797. Acesso em: 29 abril 2025.

\bibitem{gregorio2020volatilidade}
GREGORIO, R. Volatilidade do Ibovespa em crises recentes: uma análise estatística. \textit{Revista Brasileira de Finanças}, v. 18, n. 1, p. 75--98, 2020. Disponível em: https://bibliotecadigital.fgv.br/ojs/index.php/rbfin/article/view/83258. Acesso em: 29 abril 2025.

\bibitem{ilmanen2022investing}
ILMANEN, A. \textit{Investing amid low expected returns: making the most when markets offer the least}. Hoboken: Wiley, 2022. Disponível em: https://www.wiley.com/en-us/Investing+Amid+Low+Expected+Returns\%3A+Making+the+Most+When+Markets+Offer+the+Least-p-9781119860198. Acesso em: 13 maio 2025.

\bibitem{khan2022stock}
KHAN, M.; SHAIKH, S. Stock price analysis and forecasting using Python. \textit{Journal of Financial Innovation}, v. 7, n. 2, p. 25--37, 2022. Disponível em: https://papers.ssrn.com/sol3/papers.cfm?abstract\_id=4051293. Acesso em: 1 maio 2025.

\bibitem{maillard2010properties}
MAILLARD, S.; RONCALLI, T.; TEILETCHE, J. On the properties of equally-weighted risk contributions portfolios. \textit{Journal of Portfolio Management}, v. 36, n. 4, p. 60--70, 2010. Disponível em: https://jpm.pm-research.com/content/36/4/60. Acesso em: 29 abril 2025.

\bibitem{markowitz1952portfolio}
MARKOWITZ, H. Portfolio selection. \textit{Journal of Finance}, v. 7, n. 1, p. 77--91, 1952. Disponível em: https://www.jstor.org/stable/2975974. Acesso em: 29 abril 2025.

\bibitem{mckinney2017python}
MCKINNEY, W. \textit{Python for Data Analysis: data wrangling with Pandas, NumPy, and IPython}. 2. ed. Sebastopol, CA: O'Reilly Media, 2017.

\bibitem{mcfedries2022python}
MCFEDRIES, P. \textit{Python QuickStart Guide: the simplified beginner's guide to Python programming}. Pittsburgh: ClydeBank Media, 2022.

\bibitem{michalak2024equal}
MICHALAK, T.; PAKUŁA, M.; PŁOŃSKA, A. Equal Weight versus Hierarchical Risk Parity Portfolios: a comparative study. \textit{Financial Research Letters}, v. 54, art. 104007, 2024. Disponível em: https://www.sciencedirect.com/science/article/pii/S1544612323003879. Acesso em: 29 abril 2025.

\bibitem{oliphant2015guide}
OLIPHANT, T. \textit{Guide to NumPy}. 2. ed. Charleston, SC: CreateSpace, 2015.

\bibitem{palit2024study}
PALIT, R.; PRYBUTOK, V. R. A study of Hierarchical Risk Parity in portfolio construction. \textit{Finance \& Economics Review}, v. 6, n. 1, p. 1--12, 2024. Disponível em: https://doi.org/10.38157/fer.v6i1.609. Acesso em: 29 abril 2025.

\bibitem{pereira2021impacto}
PEREIRA, C. M.; COLOMBO, C.; FIGUEIREDO, M. V. Impacto de choques políticos no mercado acionário brasileiro: uma análise de eventos. \textit{Revista de Administração Contemporânea}, v. 25, n. 5, p. 743--764, 2021. Disponível em: https://rac.anpad.org.br/index.php/rac/article/view/1617. Acesso em: 29 abril 2025.

\end{thebibliography}

\end{document}