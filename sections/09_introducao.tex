% ==================================================================
% 1 INTRODUÇÃO
% ==================================================================

\chapter{INTRODUÇÃO}

A alocação de ativos é amplamente reconhecida como um dos principais determinantes do desempenho de carteiras de investimento. Estudos clássicos, como o de \cite{brinson1986determinants}, indicam que mais de 90\% da variância do retorno de uma carteira pode ser explicada por decisões de alocação estratégica de ativos, superando o impacto da seleção individual de ativos ou do timing de mercado.

Nesse contexto \cite{markowitz1952portfolio} propôs o modelo de Média-Variância, que representa um marco na teoria moderna de portfólios ao formalizar matematicamente a relação entre risco e retorno esperado. Sua abordagem busca identificar combinações eficientes de ativos que maximizem o retorno esperado para um dado nível de risco, ou minimizem o risco para um retorno desejado.

Em ambientes caracterizados por elevada volatilidade e incerteza, como frequentemente ocorre em mercados emergentes, a definição de uma estratégia de alocação eficiente torna-se ainda mais desafiadora, exigindo metodologias que consigam lidar com instabilidade, correlações variáveis e estimativas imperfeitas de risco e retorno \cite{ilmanen2022investing}.

Entre as metodologias mais conhecidas e aplicadas na literatura acadêmica e no mercado estão o modelo de Média-Variância, proposto por Markowitz, a estratégia de alocação por pesos iguais (Equal Weight) e a metodologia de paridade de risco (Risk Parity). Cada uma dessas abordagens apresenta características específicas, vantagens próprias e limitações que precisam ser cuidadosamente analisadas em ambientes voláteis.

O modelo de \cite{markowitz1952portfolio} revolucionou a teoria financeira ao formalizar matematicamente a construção de carteiras eficientes, baseando-se na relação entre risco e retorno esperado. Seu principal objetivo é identificar a combinação ótima de ativos que maximize o retorno esperado para um nível específico de risco ou minimize o risco para determinado nível de retorno. Entretanto, esse modelo assume condições como a normalidade dos retornos dos ativos e a estabilidade das estimativas utilizadas, premissas que nem sempre se verificam na prática, especialmente em períodos de alta volatilidade ou crises financeiras \cite{michalak2024equal}.

Como alternativa de implementação mais simples, a estratégia Equal Weight distribui o capital igualmente entre todos os ativos selecionados na carteira, sem a necessidade de previsões complexas. Essa abordagem demonstra, em muitos estudos, ser bastante robusta em cenários de alta incerteza, apresentando desempenho comparável, ou até superior, a estratégias de otimização mais sofisticadas, especialmente em análises fora da amostra \cite{demiguel2009optimal}. Por outro lado, sua simplicidade implica limitações, pois ignora características fundamentais dos ativos, como volatilidade e correlação, o que pode levar a concentrações de risco inadvertidas.

A metodologia de Risk Parity, por sua vez, busca uma distribuição mais equilibrada do risco total da carteira, atribuindo menores pesos a ativos mais voláteis e maiores pesos a ativos menos voláteis. Tal abordagem vem ganhando destaque nos últimos anos por produzir carteiras mais estáveis e menos suscetíveis a erros de estimativa, com desempenho sólido em diferentes cenários econômicos \cite{maillard2010properties,palit2024study}.

No cenário brasileiro, o período compreendido entre 2016 e 2019 foi marcado por alta volatilidade no mercado acionário, com o desvio-padrão anualizado dos retornos do Ibovespa oscilando entre 20\% e 25\% \cite{gregorio2020volatilidade}. Particularmente, os anos de 2018 e 2019 coincidiram com um contexto de incerteza política e financeira, principalmente em função das eleições presidenciais e das alterações no ambiente econômico subsequente. Estudos de \cite{pereira2021impacto} mostraram que choques políticos influenciaram diretamente os retornos de ações brasileiras, especialmente de empresas com vínculos governamentais, enquanto \cite{carnahan2020electoral} comprovaram que eleições em mercados emergentes tendem a aumentar significativamente a volatilidade dos ativos no curto prazo.

Diante desse contexto de instabilidade e alta incerteza, o presente trabalho propõe uma análise comparativa entre as três estratégias de alocação de carteira --- Markowitz, Equal Weight e Risk Parity --- utilizando dados de ativos negociados na B3 no período de 2018 a 2019. A comparação do desempenho será realizada com base em duas métricas amplamente reconhecidas na literatura financeira: o Índice de Sharpe, que avalia o retorno ajustado ao risco total da carteira, e o Sortino Ratio, que considera apenas os riscos de perdas.

Com essa abordagem, pretende-se contribuir para a identificação de estratégias de alocação mais eficientes no contexto brasileiro, gerando insights relevantes tanto para investidores quanto para gestores de recursos que buscam maximizar o retorno ajustado ao risco em ambientes de elevada volatilidade e imprevisibilidade.

\section{OBJETIVO GERAL}

Analisar comparativamente o desempenho das estratégias de alocação de carteira Markowitz, Equal Weight e Risk Parity no mercado brasileiro, utilizando dados de ativos da B3 entre 2018 e 2019, com base nos indicadores Índice de Sharpe e Sortino Ratio, a fim de identificar a estratégia mais eficiente em termos de retorno ajustado ao risco.

\section{OBJETIVOS ESPECÍFICOS}

\begin{itemize}
    \item Selecionar uma amostra de aproximadamente 10 ações da B3, considerando critérios de liquidez, representatividade setorial e capitalização de mercado.
    
    \item Calcular os retornos históricos dos ativos selecionados, estimar parâmetros como médias, volatilidades e covariâncias dos retornos.
    
    \item Implementar as três estratégias de alocação (Markowitz, Equal Weight e Risk Parity), programaticamente, por meio de ferramentas computacionais.
    
    \item Realizar o rebalanceamento semestral das carteiras durante o período de 2018 a 2019.
    
    \item Calcular os Índices de Sharpe e Sortino para cada carteira e para o período consolidado.
    
    \item Comparar os desempenhos obtidos, avaliando a eficiência de cada estratégia em ambientes de alta volatilidade e instabilidade política.
\end{itemize}

\section{JUSTIFICATIVA}

A escolha da estratégia de alocação de ativos é um dos fatores mais determinantes para o desempenho de carteiras de investimento, especialmente em mercados caracterizados por alta volatilidade e incerteza, como o brasileiro. Nesse cenário, compreender a eficácia das diferentes metodologias de construção de portfólios torna-se crucial para investidores e gestores que buscam maximizar retornos e mitigar riscos.

O modelo de Markowitz, embora amplamente consolidado, apresenta limitações práticas, sobretudo pela dependência de estimativas instáveis em momentos de elevada volatilidade. A estratégia Equal Weight, por sua vez, oferece simplicidade operacional e robustez frente a erros de previsão, mas desconsidera características fundamentais dos ativos, como a volatilidade individual. Já a metodologia de Risk Parity surge como uma alternativa promissora, ao equilibrar a contribuição de risco entre os ativos, proporcionando maior estabilidade às carteiras.

No contexto do mercado brasileiro, o período de 2018 a 2019 foi particularmente marcado por oscilações intensas, impulsionadas por fatores políticos e econômicos. Segundo \cite{gregorio2020volatilidade}, a volatilidade anualizada do Ibovespa em 2018 alcançou aproximadamente 25\%, superando a média histórica observada em períodos de maior estabilidade. Esse ambiente volátil oferece uma oportunidade ideal para investigar como diferentes abordagens de alocação se comportam frente a choques exógenos e incertezas sistêmicas.

A presente pesquisa justifica-se, portanto, pela necessidade de avaliar, com dados reais e métricas consolidadas --- como o Índice de Sharpe e o Sortino Ratio ---, qual das três estratégias de alocação proporciona melhor relação risco-retorno em um mercado emergente. Além disso, os resultados obtidos podem oferecer insights relevantes para a prática de gestão de portfólios, especialmente por fundos multimercado, gestoras de ativos e investidores individuais que operam em ambientes instáveis.

Por fim, ao abordar simultaneamente indicadores clássicos e medidas voltadas ao risco de perdas, o estudo busca apresentar uma avaliação mais completa da eficiência das estratégias analisadas, contribuindo tanto para o avanço da literatura quanto para a tomada de decisão no mercado financeiro.