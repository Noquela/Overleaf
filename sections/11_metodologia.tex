% ==================================================================
% 3 METODOLOGIA
% ==================================================================

\chapter{METODOLOGIA}

\section{TIPO DE PESQUISA E ESTRATÉGIA METODOLÓGICA}

Este estudo é de natureza quantitativa, descritiva e comparativa, com foco na avaliação do desempenho de diferentes estratégias de alocação de ativos financeiros. A pesquisa adota uma abordagem empírica, utilizando dados históricos do mercado financeiro brasileiro para a construção e análise das carteiras.

\section{PERÍODO E AMBIENTE DE ESTUDO}

O horizonte temporal da análise compreende o período de janeiro de 2018 a dezembro de 2019, um momento de alta volatilidade e instabilidade política no Brasil. O ambiente de estudo é a B3 -- Brasil Bolsa Balcão, principal bolsa de valores brasileira.

\section{SELEÇÃO DOS ATIVOS}

A seleção dos ativos da amostra será baseada nos seguintes critérios:

\begin{itemize}
    \item \textbf{Alta liquidez histórica:} Volume médio diário de negociação superior a R\$ 50 milhões, calculado com base no histórico de janeiro de 2016 a dezembro de 2017;
    
    \item \textbf{Diversificação setorial:} Inclusão de ações de diferentes setores da economia;
    
    \item \textbf{Capitalização de mercado:} Preferência por empresas de maior valor de mercado, geralmente pertencentes ao índice Ibovespa.
\end{itemize}

Essa metodologia de seleção busca garantir que os ativos escolhidos sejam representativos, líquidos e compatíveis com práticas profissionais de alocação de portfólio, evitando viés de seleção retrospectivo.

\section{COLETA E TRATAMENTO DOS DADOS}

Os dados históricos de preços ajustados dos ativos serão coletados utilizando APIs financeiras (como Yahoo Finance), via Python.

O tratamento dos dados inclui:

\begin{itemize}
    \item Cálculo dos retornos mensais e anualizados;
    \item Estimativa das volatilidades individuais;
    \item Construção da matriz de covariância entre os retornos dos ativos;
    \item Definição da taxa livre de risco como o CDI médio anualizado do período.
\end{itemize}

\section{CONSTRUÇÃO DAS CARTEIRAS}

Serão implementadas três estratégias de alocação:

\subsection{Markowitz (Média-Variância)}
Otimização para maximizar o Índice de Sharpe, com restrições de soma dos pesos igual a 1 e ausência de vendas a descoberto.

\subsection{Equal Weight}
Alocação igualitária do capital entre os ativos.

\subsection{Risk Parity}
Alocação baseada na contribuição igual de risco de cada ativo, utilizando a fórmula:

\begin{equation}
w_i = \frac{(1/\sigma_i)}{\sum_{j=1}^{n}(1/\sigma_j)}
\end{equation}

As carteiras serão construídas usando a linguagem Python, com bibliotecas como pandas, NumPy e cvxpy.

\section{REBALANCEAMENTO DAS CARTEIRAS}

O rebalanceamento das carteiras será realizado de forma semestral, nos meses de janeiro e julho, utilizando as novas estimativas de retornos e covariâncias disponíveis em cada momento.

Custos de transação, impostos e slippage não serão considerados, representando uma limitação reconhecida da pesquisa.

\section{AVALIAÇÃO DE DESEMPENHO}

O desempenho das carteiras será avaliado por:

\begin{itemize}
    \item \textbf{Índice de Sharpe:} Avaliação do retorno excedente ajustado pela volatilidade total.
    \item \textbf{Sortino Ratio:} Avaliação do retorno excedente ajustado apenas pelo risco de perdas (volatilidade negativa).
\end{itemize}

As métricas serão calculadas:
\begin{itemize}
    \item Para cada semestre individualmente;
    \item E para o período consolidado 2018--2019.
\end{itemize}

\section{FLUXO DE PROCESSAMENTO E ANÁLISE DE RESULTADOS}

Este trabalho adota um fluxo integrado que vai da extração dos dados até a análise comparativa dos resultados, organizado em quatro fases principais:

\subsection{Coleta e Tratamento de Dados}

Inicialmente, extraem-se automaticamente as cotações ajustadas diárias dos ativos selecionados (B3, 2018--2019). Essas séries passam por validação --- remoção de duplicatas, interpolação de eventuais lacunas e filtragem de outliers ---, garantindo uma base limpa e contínua. Em sequência, convertem-se as variações diárias em retornos mensais compostos, produzindo uma tabela onde cada linha representa o retorno mensal de cada ativo.

\subsection{Cálculo de Insumos para Alocação}

A partir dos retornos mensais, calcula-se o vetor de retorno médio ($\mu$) e a matriz de covariância ($\Sigma$). Esses insumos são imediatamente utilizados em cada data de rebalanceamento (1º de janeiro e 1º de julho), garantindo que a alocação reflita apenas as informações disponíveis até aquele ponto.

\subsection{Construção de Carteiras e Extração de Pesos}

\begin{itemize}
    \item \textbf{Markowitz (Sharpe máximo):} otimiza-se a relação entre retorno excedente e risco total, sujeita a soma unitária e não negatividade dos pesos.
    
    \item \textbf{Equal Weight:} atribui-se peso igual a todos os ativos (1/N), servindo como benchmark simples.
    
    \item \textbf{Risk Parity:} ajusta-se iterativamente os pesos até que cada ativo contribua igualmente para o risco total da carteira (produto do peso pela volatilidade marginal).
\end{itemize}

Esses vetores de pesos passam a compor três séries de evolução de portfólio.

\subsection{Cálculo de Indicadores e Análise Comparativa}

As séries de retorno de cada carteira alimentam o cálculo das principais métricas:

\begin{itemize}
    \item Sharpe Ratio (retorno excedente $\div$ volatilidade total);
    \item Sortino Ratio (retorno excedente $\div$ volatilidade negativa);
    \item Drawdown Máximo (pior queda acumulada);
    \item Volatilidade Realizada (desvio-padrão efetivo dos retornos).
\end{itemize}

Para cada semestre e para o acumulado de 2018--2019, organizam-se tabelas comparativas, seguidas de:

\begin{itemize}
    \item Gráficos de linha mostrando a evolução do capital investido, permitindo visualizar divergências de performance;
    \item Box-plots das distribuições mensais de retorno, evidenciando dispersão e assimetrias;
    \item Heatmaps de correlação periódicos para observar mudanças no relacionamento entre ativos e seu impacto nas carteiras.
\end{itemize}

Na redação do TCC, cada indicador será apresentado em seção própria de Resultados, com interpretação centrada em:

\begin{itemize}
    \item Diferenças de eficiência entre as três estratégias em ambientes de alta volatilidade;
    \item Robustez dos resultados a diferentes janelas de rebalanceamento;
    \item Significância prática das variações de Sharpe e Sortino, discutindo potenciais custos de transação e limitações não consideradas.
\end{itemize}

Esse fluxo completo --- da base de preços até a síntese crítica dos resultados --- serve de alicerce para a discussão final, embasando recomendações e apontando direções para pesquisas futuras.

\section{FLUXOGRAMA METODOLÓGICO}

\begin{figure}[h]
\centering
\caption{Fluxograma da Metodologia}
% Adicionar figura quando disponível
\textit{Fonte: Elaborado pelo autor.}
\label{fig:fluxograma_metodologia}
\end{figure}

\begin{table}[h]
\centering
\caption{Etapas da Pesquisa e Ferramentas Utilizadas}
\begin{tabular}{|p{4cm}|p{4cm}|p{4cm}|}
\hline
\textbf{Etapa} & \textbf{Descrição} & \textbf{Ferramentas} \\
\hline
Coleta de Dados & Extração de preços históricos da B3 & Python, yfinance \\
\hline
Tratamento de Dados & Cálculo de retornos e estatísticas & pandas, NumPy \\
\hline
Construção de Carteiras & Implementação das três estratégias & cvxpy, pandas \\
\hline
Análise de Desempenho & Cálculo de métricas e comparação & NumPy, matplotlib \\
\hline
Visualização & Gráficos e tabelas comparativas & matplotlib, seaborn \\
\hline
\end{tabular}
\label{tab:etapas_pesquisa}
\end{table}

\section{LIMITAÇÕES DO ESTUDO}

Apesar do rigor metodológico adotado, este estudo apresenta algumas limitações que devem ser consideradas na análise dos resultados. Em primeiro lugar, não foram incorporados custos de transação, taxas, slippage e tributação nas operações de compra e venda dos ativos, o que pode gerar divergências entre os retornos simulados e os efetivamente obtidos na prática. Além disso, a utilização de séries históricas de retornos pressupõe que padrões passados se mantenham representativos para o futuro, o que pode não se confirmar em mercados sujeitos a choques exógenos e mudanças estruturais. Outra limitação refere-se à escolha de apenas três estratégias de alocação, desconsiderando alternativas mais recentes como o modelo de Hierarchical Risk Parity ou abordagens baseadas em Machine Learning, que poderiam trazer novas perspectivas. Por fim, a definição da taxa livre de risco como o CDI médio anualizado simplifica a realidade de investimentos no Brasil, que apresenta múltiplos instrumentos de renda fixa com diferentes graus de risco e liquidez. Tais limitações, embora não invalidem os resultados, indicam caminhos para aprofundamentos em pesquisas futuras.