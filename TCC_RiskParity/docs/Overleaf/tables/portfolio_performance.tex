% Tabela gerada automaticamente pelo Python
\begin{table}[H]
\centering
\caption{Performance Consolidada das Carteiras (2018-2019)}
\begin{tabular}{|l|r|r|r|r|r|r|}
\hline
\textbf{Estratégia} & \textbf{Retorno} & \textbf{Volatilidade} & \textbf{Sharpe} & \textbf{Sortino} & \textbf{Max} & \textbf{VaR} \\
& \textbf{Anual (\%)} & \textbf{Anual (\%)} & \textbf{Ratio} & \textbf{Ratio} & \textbf{Drawdown} & \textbf{95\%} \\
\hline
Markowitz & 1.8% & 28.0% & -0.17 & -0.20 & -35.6% & -179.7% \\
\hline
Equal Weight & -6.4% & 23.3% & -0.55 & -0.69 & -34.8% & -168.2% \\
\hline
Risk Parity & -5.6% & 22.8% & -0.53 & -0.64 & -33.9% & -161.4% \\
\hline
\textbf{Ibovespa} & \textbf{8,1\%} & \textbf{28,3\%} & \textbf{0,06} & \textbf{0,09} & \textbf{-41,2\%} & \textbf{-42,5\%} \\
\hline
\end{tabular}
\textit{Fonte: Elaborado pelo autor utilizando Python com dados da Economatica. Taxa livre de risco: 6,5\% a.a.}
\label{tab:portfolio_performance}
\end{table}
