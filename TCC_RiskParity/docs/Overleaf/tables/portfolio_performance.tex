% Tabela gerada automaticamente pelo Python
\begin{table}[H]
\centering
\caption{Performance Consolidada das Carteiras (2018-2019)}
\begin{tabular}{|l|r|r|r|r|r|r|}
\hline
\textbf{Estratégia} & \textbf{Retorno} & \textbf{Volatilidade} & \textbf{Sharpe} & \textbf{Sortino} & \textbf{Max} & \textbf{VaR} \\
& \textbf{Anual (\%)} & \textbf{Anual (\%)} & \textbf{Ratio} & \textbf{Ratio} & \textbf{Drawdown} & \textbf{95\%} \\
\hline
Markowitz & 32.61% & 16.67% & 2.38 & 11.24 & -14.6% & -4.2% \\
\hline
Equal Weight & 35.93% & 22.04% & 1.86 & 9.68 & -19.8% & -5.8% \\
\hline
Risk Parity & 29.53% & 18.45% & 1.84 & 12.33 & -16.2% & -4.9% \\
\hline
\textbf{Ibovespa} & \textbf{8,1\%} & \textbf{28,3\%} & \textbf{0,06} & \textbf{0,09} & \textbf{-41,2\%} & \textbf{-42,5\%} \\
\hline
\end{tabular}

\textit{Fonte: Elaborado pelo autor utilizando Python com dados da Economatica. Taxa livre de risco: 6,195\% a.a.}
\label{tab:portfolio_performance}
\end{table}
